% Generated by GrindEQ Word-to-LaTeX 
\documentclass{article} % use \documentstyle for old LaTeX compilers

\usepackage[utf8]{inputenc} % 'cp1252'-Western, 'cp1251'-Cyrillic, etc.
\usepackage[english]{babel} % 'french', 'german', 'spanish', 'danish', etc.
\usepackage{amsmath}
\usepackage{amssymb}
\usepackage{txfonts}
\usepackage{mathdots}
\usepackage[classicReIm]{kpfonts}
\usepackage[pdftex]{graphicx}

% You can include more LaTeX packages here 


\begin{document}

%\selectlanguage{english} % remove comment delimiter ('%') and select language if required


\noindent Universidad Mariano G\'{a}lvez de Guatemala

\noindent ingenier\'{i}a en sistemas

\noindent Seguridad y Auditoria de sistemas

\noindent 

\noindent 

\noindent 

\noindent 

\noindent 

\noindent 

\noindent 

\noindent 

\noindent 

\noindent 

\noindent 

\noindent 

\noindent 

\noindent 

\noindent 

\noindent 

\noindent 

\noindent 

\noindent 

\noindent Implementaci\'{o}n de pfSense con red en AWS

\noindent 

\noindent 

\noindent 

\noindent 

\noindent 

\noindent 

\noindent 

\noindent 

\noindent 

\noindent 

\noindent 

\noindent 

\noindent 

\noindent 

\noindent 

\noindent 

\noindent 

\noindent 

\noindent 

\noindent 

\noindent 

\noindent Eliezer Osbaldo M\'{e}ndez Valle {\textbar} 7690-14-9683 {\textbar} Secci\'{o}n A

\noindent Geoffrey Estiven Hern\'{a}ndez Franco {\textbar} 7690-14-3807

\noindent Sergio Alexander Tzalam Lopez {\textbar} 7690-16-3621

\noindent Lo primero para crear nuestra implementaci\'{o}n es necesario crear VPC

\noindent 

\noindent \includegraphics*[width=4.51in, height=3.32in, trim=0.00in 0.14in 3.06in 0.11in]{image1}

\noindent 

\noindent 

\noindent Procedemos a ingresar los datos, la cual creamos como pesense\_vpc con los siguientes datos.

\noindent 

\noindent \includegraphics*[width=3.05in, height=3.70in, trim=0.03in 0.17in 5.15in 0.00in]{image2}

\noindent 

\noindent Procedemos a dar clic en crear vpc

\noindent 

\noindent Lo siguiente es crear el Peering que va a apuntar a nuestro end-point. 

\noindent 

\noindent \includegraphics*[width=6.66in, height=2.21in, trim=0.00in 2.20in 2.93in 0.14in]{image3}

\noindent 

\noindent 

\noindent 

\noindent 

\noindent Ingresamos el nombre, ingresamos el BGP ASN , que ser\'{a} el sistema aut\'{o}nomo del Peer, se debe de tomar en cuenta que la VPN debe de ser ruteable. Para la IP se ingresa nuestra IP p\'{u}blica. 

\noindent 

\noindent 

\noindent \includegraphics*[width=6.74in, height=3.38in, trim=0.00in 2.36in 5.55in 0.00in]{image4}

\noindent 

\noindent 

\noindent 

\noindent 

\noindent 

\noindent 

\noindent 

\noindent Luego procedemos a crear las Subnets, como se pueden visualziar ya hay registradas las que viene de forma predefinida en aw. 

\noindent 

\noindent \includegraphics*[width=6.45in, height=1.03in, trim=0.62in 2.22in 0.09in 0.14in]{image5}

\noindent 

\noindent Procedemos a crear la nueva subnet. 

\noindent 

\noindent \includegraphics*[width=5.41in, height=1.89in, trim=7.04in 3.87in 0.00in 0.22in]{image6}

\noindent 

\noindent Ingresamos los datos, el nombre, le indicamos el rango de IP, y luego clic en Crear Subnet. 

\noindent 

\noindent \includegraphics*[width=3.35in, height=3.89in, trim=0.00in 0.00in 5.58in 0.17in]{image7}

\noindent Como podremos ver nuestra subnet ya fue creada correctamente

\noindent 

\noindent \includegraphics*[width=4.78in, height=2.04in, trim=1.29in 4.28in 7.67in 0.12in]{image8}

\noindent 

\noindent Luego lo que prosigue es crear la Gateway privado, este sirve para que nuestro trafico salga a nuestro Gateway privado para luego adjuntarlo a nuestro VPC

\noindent 

\noindent 

\noindent 

\noindent \includegraphics*[width=5.75in, height=2.04in, trim=0.00in 6.50in 12.29in 0.00in]{image9}

\noindent 

\noindent Le damos un nombre, y seleccionamos Amazon default ASN, que lo que va a hacer es darle el sistema aut\'{o}nomo de amazon

\noindent 

\noindent \includegraphics*[width=6.21in, height=1.75in, trim=0.00in 3.32in 4.49in 0.00in]{image10}

\noindent 

\noindent 

\noindent 

\noindent 

\noindent Procedemos a crearlo

\noindent 

\noindent \includegraphics*[width=5.45in, height=1.72in, trim=0.00in 4.23in 7.67in 0.00in]{image11}

\noindent Luego ya nos aparecer\'{a} en el listado, lo seleccionamos y lo asignaremos al VPC, clic en Actions luego en Attach to VPC

\noindent 

\noindent 

\noindent \includegraphics*[width=5.72in, height=0.91in, trim=0.96in 4.21in 4.81in 0.12in]{image12}

\noindent Luego selecionamos la vpc que ya habiamos creado anteriormente  

\noindent 

\noindent \includegraphics*[width=5.60in, height=1.97in, trim=0.00in 2.94in 4.79in 0.00in]{image13}

\noindent 

\noindent presionamos Yes, Attach. 

\noindent 

\noindent \includegraphics*[width=5.56in, height=1.59in, trim=0.00in 3.18in 4.68in 0.00in]{image14}

\noindent 

\noindent 

\noindent Como podremos ver ya nos aparece asignada a nuestra VPC ya solo es de esperara para que esta se a\~{n}ada. 

\noindent \includegraphics*[width=6.75in, height=1.21in, trim=0.00in 4.78in 5.72in 0.00in]{image15}

\noindent 

\noindent Lo que prosigue es realizar la degeneraci\'{o}n de BGP sobre las routing tables. 

\noindent 

\noindent \includegraphics*[width=5.92in, height=1.21in, trim=0.00in 2.80in 2.85in 0.00in]{image16}

\noindent 

\noindent 

\noindent \includegraphics*[width=6.04in, height=1.67in, trim=0.68in 1.49in 0.10in 0.10in]{image17}

\noindent 

\noindent Procedemos a habilitar la propagaci\'{o}n

\noindent 

\noindent \includegraphics*[width=3.81in, height=2.72in, trim=0.00in 2.13in 6.39in 0.00in]{image18}

\noindent Como podemos ver ahora en propagaci\'{o}n ya nos aparece YES, y como podemos ver va a propagar la red que se encuentre asociada al pr\'{i}vate Gateway. 

\noindent 

\noindent \includegraphics*[width=5.73in, height=3.19in, trim=1.07in 1.37in 3.09in 0.14in]{image19}

\noindent 

\noindent Luego procedemos a crear la VPN site to site. 

\noindent 

\noindent \includegraphics*[width=6.21in, height=1.29in, trim=0.00in 2.72in 2.58in 0.00in]{image20}

\noindent 

\noindent 

\noindent 

\noindent 

\noindent 

\noindent 

\noindent 

\noindent 

\noindent 

\noindent 

\noindent 

\noindent 

\noindent 

\noindent 

\noindent 

\noindent 

\noindent 

\noindent Colocaremos los parametros minimos solicitados para crearla con la getway y privatw gatewat , los parametros que se dejaron en blanco es para que estas tomen los datos predefinidos por defecto. 

\noindent 

\noindent \includegraphics*[width=6.05in, height=6.54in, trim=0.00in 0.21in 7.84in 0.00in]{image21}

\noindent 

\noindent 

\noindent 

\noindent 

\noindent 

\noindent 

\noindent 

\noindent 

\noindent 

\noindent El inside IPv4 se dejar\'{a} en blanco para que tome los datos predefinidos de AWS. Tomara un segmento 30 del segmento 254.0.0/16 esto para que no pueda ser ruteado. 

\noindent 

\noindent \includegraphics*[width=5.72in, height=6.34in, trim=0.00in 0.17in 7.83in 0.00in]{image22}

\noindent 

\noindent 

\noindent 

\noindent \includegraphics*[width=5.71in, height=6.32in, trim=0.00in 0.25in 8.16in 0.00in]{image23}

\noindent 

\noindent 

\noindent \includegraphics*[width=4.63in, height=3.96in, trim=0.00in 1.49in 6.56in 0.00in]{image24}

\noindent 

\noindent 

\noindent Procedemos a crearla y ya se nos mostrara en el listado de VPN connection. 

\noindent 

\noindent \includegraphics*[width=6.51in, height=0.91in, trim=0.00in 2.80in 1.32in 0.00in]{image25}

\noindent 

\noindent 

\noindent Ahora en pfsense ingresaremos la configuraciones en relaci\'{o}n a las realizadas a AWS.

\noindent 

\noindent \includegraphics*[width=6.13in, height=6.11in]{image26}

\noindent 

\noindent Para realizar las configuraciones AWS nos facilita un archivo donde vienen todos las indicaciones para realizar la configuraci\'{o}n en pfSense.  Nos dirigimos a site-to-site VPN conection y damos clic en Download configuration. 

\noindent 

\noindent 

\noindent \includegraphics*[width=4.34in, height=1.36in, trim=0.00in 4.76in 8.62in 0.00in]{image27}

\noindent 

\noindent Se proceder\'{a} a descargar el archivo. 

\noindent 

\noindent \includegraphics*[width=5.07in, height=2.00in, trim=0.00in 0.90in 0.98in 0.00in]{image28}

\noindent 

\noindent Como podremos ver dentro del archivo vienen todos los datos necesarios para configurar nuestro pfSense. 

\noindent 

\noindent 

\noindent \includegraphics*[width=6.04in, height=3.96in, trim=0.00in 0.00in 0.00in 0.08in]{image29}

\noindent 

\noindent Podremos ver en el documento las dos interfaces con que se creo el tunnel 

\noindent \includegraphics*[width=4.71in, height=1.81in]{image30}

\noindent 

\noindent 

\noindent 

\noindent Antes de ingresar los datos de AWS tenemos que configurar pfSense de la siguiente manera.

\noindent 

\noindent 

\noindent \includegraphics*[width=5.43in, height=5.29in, trim=2.74in 0.00in 2.88in 0.00in]{image31}

\noindent 

\noindent 

\noindent \includegraphics*[width=5.53in, height=5.27in, trim=2.69in 0.00in 2.78in 0.00in]{image32}

\noindent 

\noindent 

\noindent 

\noindent 

\noindent 

\noindent 

\noindent \includegraphics*[width=5.11in, height=4.75in, trim=2.39in 0.00in 2.20in 0.00in]{image33}

\noindent 

\noindent 

\noindent \includegraphics*[width=4.59in, height=4.39in, trim=2.13in 0.00in 2.21in 0.00in]{image34}

\noindent 

\noindent \includegraphics*[width=4.43in, height=4.25in, trim=1.65in 0.00in 1.51in 0.00in]{image35}

\noindent 

\noindent 

\noindent \includegraphics*[width=4.99in, height=4.49in, trim=1.96in 0.00in 1.70in 0.00in]{image36}

\noindent 

\noindent Procedemos a dar clic en guardar. 

\noindent 

\noindent 

\noindent Y luego en aplicar cambios.  

\noindent \includegraphics*[width=5.38in, height=1.62in, trim=1.32in 1.73in 0.98in 0.62in]{image37}

\noindent 

\noindent \includegraphics*[width=4.99in, height=2.26in, trim=0.63in 0.54in 0.51in 0.29in]{image38}

\noindent 

\noindent 

\noindent Luego procedemos a complementar los datos con los del archivo descargarlo en aws. 

\noindent 

\noindent 

\noindent Nos dirimos a interface Assignament. 

\noindent \includegraphics*[width=5.22in, height=2.99in, trim=0.42in 0.78in 0.49in 0.27in]{image39}

\noindent 

\noindent \includegraphics*[width=5.22in, height=5.03in, trim=1.04in 0.00in 1.27in 0.00in]{image40}

\noindent 

\noindent 

\noindent \includegraphics*[width=6.01in, height=3.71in, trim=0.32in 0.70in 0.49in 0.34in]{image41}

\noindent 

\noindent 

\noindent \includegraphics*[width=6.12in, height=3.85in, trim=0.48in 0.46in 0.48in 0.41in]{image42}

\noindent 

\noindent 

\noindent \includegraphics*[width=6.41in, height=5.81in, trim=1.69in 0.00in 0.87in 0.40in]{image43}

\noindent 

\noindent Procedemos a validar el estatus de IPsec

\noindent \includegraphics*[width=1.39in, height=2.55in, trim=4.95in 3.87in 3.24in 0.00in]{image44}

\noindent 

\noindent \includegraphics*[width=6.00in, height=1.57in, trim=0.85in 1.31in 0.85in 0.39in]{image45}

\noindent 

\noindent 

\noindent \includegraphics*[width=6.22in, height=2.09in, trim=0.42in 0.56in 0.09in 0.33in]{image46}

\noindent 

\noindent Damos clic en SADs y podremos ver que ya esta tanto recibiendo como enviando trafico de AWS.  

\noindent 

\noindent 

\noindent \includegraphics*[width=6.14in, height=1.71in, trim=1.04in 1.78in 1.11in 0.38in]{image47}


\end{document}

